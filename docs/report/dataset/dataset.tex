%!TEX program = xelatex
%!TEX spellcheck = en_GB
\documentclass[final]{report}
% Include all project wide packages here.
%\usepackage{fullpage}
\usepackage[a4paper,margin=2.5cm,top=2cm]{geometry}
\usepackage{polyglossia}
\setmainlanguage{english}
\usepackage{csquotes}
\usepackage{graphicx}
\usepackage{pdfpages}
\usepackage{caption}
\usepackage[list=true]{subcaption}
\usepackage{float}
\usepackage{standalone}
\usepackage{import}
\usepackage{tocloft}
\usepackage{wrapfig}
\usepackage{authblk}
\usepackage{array}
\usepackage{booktabs}
\usepackage[title,titletoc]{appendix}
\usepackage{fontspec}
\usepackage{pgfplots}
\usepackage{tikz}
\usepackage[binary-units=true]{siunitx}
\usepackage{units}
\usepackage{amsmath}
\usepackage{mathtools}
\usepackage{unicode-math}
\usepackage{rotating}
\usepackage{titlesec}
\usepackage{titletoc}
\usepackage{blindtext}
\usepackage{color}
\usepackage{enumitem}
\usepackage{tabularx}
\usepackage{titling}
\usepackage[%
siunitx,
fulldiodes,
europeanvoltages,
europeancurrents,
europeanresistors,
americaninductors,
smartlabels]{circuitikz}

\newcommand{\matlab}{{\textsc{matlab }}}

\usetikzlibrary{calc}
\usetikzlibrary{positioning}
\usetikzlibrary{automata}
\usetikzlibrary{arrows.meta}

\tikzstyle{every state}=[fill=tu-cyan,align=center,draw=black,line width=1pt,node distance=3cm,minimum width = 1.8cm]%for FSMs casper
\tikzstyle{every initial by arrow}=[initial text={Reset}]
\newcommand{\setpathasarrows}{\tikzstyle{every path}=[auto,line width=1.5pt,line cap=round,line join=round]}

\pgfplotsset{compat=newest}
\pgfplotsset{plot coordinates/math parser=false}
\usetikzlibrary{plotmarks}
\usepgfplotslibrary{patchplots}
\newlength\figureheight
\newlength\figurewidth

\tikzset{every axis/.style={xticklabel style={align=right}}}

\usepackage[
%backend=bibtex,
backend=biber,
	texencoding=utf8,
bibencoding=utf8,
style=numeric,
citestyle=numeric,
    sortlocale=en_US,
    language=auto,
    backref=true,
    abbreviate=false,
    date=edtf,
    seconds=true
]{biblatex}


\usepackage{listings}
\newcommand{\includecode}[4][c]{\lstinputlisting[caption=#2, escapechar=, style=#1,label=#4]{#3}}
\newcommand{\superscript}[1]{\ensuremath{^{\textrm{#1}}}}
\newcommand{\subscript}[1]{\ensuremath{_{\textrm{#1}}}}


\newcommand{\chapternumber}{\thechapter}
\renewcommand{\appendixname}{Appendix}
\renewcommand{\appendixtocname}{Appendices}
\renewcommand{\appendixpagename}{Appendices}


\setlist[enumerate]{labelsep=*, leftmargin=1.5pc}
\setlist[enumerate,1]{label=\arabic*., ref=\arabic*}
\setlist[enumerate,2]{label=\arabic*.,ref=\theenumi.\arabic*}
\setlist[enumerate,3]{label=\arabic*., ref=\theenumii.\arabic*}

%\setcounter{chapter}{-1} %start chapter numbers with 0

\usepackage{xr-hyper}
\usepackage[hidelinks]{hyperref} %<--------ALTIJD ALS LAATSTE
\usepackage[nameinlink,noabbrev,capitalise]{cleveref} %<------- Clever Ref moet na hyperref
\crefname{app}{Appendix}{Appendices}
%\renewcommand{\familydefault}{\sfdefault}


\setmainfont{Myriad Pro}[Ligatures={Common,TeX}]
\setmathfont{Asana Math}
\setmonofont[Scale=0.9]{Lucida Console}
\newfontfamily\headingfont{Minion Pro}[Ligatures={Common,TeX}]


%Design colors
\definecolor{accent1}{RGB}{0,100,200}
\definecolor{accent2}{RGB}{0,50,100}
\definecolor{tu-cyan}{RGB}{0,166,214}

\newcommand{\hsp}{\hspace{20pt}}
\titleformat{\chapter}[hang]{\Huge\headingfont}{\chapternumber\hsp\textcolor{accent2}{|}\hsp}{0pt}{\Huge\headingfont}

\titleformat{name=\chapter,numberless}[hang]{\Huge\headingfont}{\hsp\textcolor{accent2}{|}\hsp}{0pt}{\Huge\headingfont}

\titleformat{\section}[block]{\LARGE\headingfont}{\arabic{chapter}.\arabic{section}}{0.4em}{}
\titleformat{\subsection}[block]{\Large\headingfont}{\arabic{chapter}.\arabic{section}.\arabic{subsection}}{0.4em}{}
\titleformat{\subsubsection}[block]{\large\headingfont}{\arabic{chapter}.\arabic{section}.\arabic{subsection}.\arabic{subsubsection}}{0.4em}{}
\renewcommand{\arraystretch}{1.2}
\renewcommand{\baselinestretch}{1.25} 

\renewcommand\cfttoctitlefont{\headingfont\Huge}
\renewcommand\cftloftitlefont{\headingfont\Huge}
\renewcommand\cftlottitlefont{\headingfont\Huge}
\setcounter{lofdepth}{2}
\setcounter{lotdepth}{2}


\setlength{\parindent}{0pt}
\setlength{\parskip}{1em}


%SIuntix settings:
%default: 0V to 10V
%custom: 0 - 10V
\sisetup{range-phrase=--}
\sisetup{range-units=single}
\DeclareSIUnit\years{years}

%For code listings
\definecolor{black}{rgb}{0,0,0}
\definecolor{browntags}{rgb}{0.65,0.1,0.1}
\definecolor{bluestrings}{rgb}{0,0,1}
\definecolor{graycomments}{rgb}{0.4,0.4,0.4}
\definecolor{redkeywords}{rgb}{1,0,0}
\definecolor{bluekeywords}{rgb}{0.13,0.13,0.8}
\definecolor{greencomments}{rgb}{0,0.5,0}
\definecolor{redstrings}{rgb}{0.9,0,0}
\definecolor{purpleidentifiers}{rgb}{0.01,0,0.01}


\lstdefinestyle{csharp}{
language=[Sharp]C,
showspaces=false,
showtabs=false,
breaklines=true,
showstringspaces=false,
breakatwhitespace=true,
escapeinside={(*@}{@*)},
columns=fullflexible,
commentstyle=\color{greencomments},
keywordstyle=\color{bluekeywords}\bfseries,
stringstyle=\color{redstrings},
identifierstyle=\color{purpleidentifiers},
basicstyle=\ttfamily\small}

\lstdefinestyle{c}{
language=C,
showspaces=false,
showtabs=false,
breaklines=true,
showstringspaces=false,
breakatwhitespace=true,
escapeinside={(*@}{@*)},
columns=fullflexible,
commentstyle=\color{greencomments},
keywordstyle=\color{bluekeywords}\bfseries,
stringstyle=\color{redstrings},
identifierstyle=\color{purpleidentifiers},
}

\lstdefinestyle{matlab}{
language=Matlab,
showspaces=false,
showtabs=false,
breaklines=true,
showstringspaces=false,
breakatwhitespace=true,
escapeinside={(*@}{@*)},
columns=fullflexible,
commentstyle=\color{greencomments},
keywordstyle=\color{bluekeywords}\bfseries,
stringstyle=\color{redstrings},
identifierstyle=\color{purpleidentifiers}
}

\lstdefinestyle{vhdl}{
language=VHDL,
showspaces=false,
showtabs=false,
breaklines=true,
showstringspaces=false,
breakatwhitespace=true,
escapeinside={(*@}{@*)},
columns=fullflexible,
commentstyle=\color{greencomments},
keywordstyle=\color{bluekeywords}\bfseries,
stringstyle=\color{redstrings},
identifierstyle=\color{purpleidentifiers}
}

\lstdefinestyle{xaml}{
language=XML,
showspaces=false,
showtabs=false,
breaklines=true,
showstringspaces=false,
breakatwhitespace=true,
escapeinside={(*@}{@*)},
columns=fullflexible,
commentstyle=\color{greencomments},
keywordstyle=\color{redkeywords},
stringstyle=\color{bluestrings},
tagstyle=\color{browntags},
morestring=[b]",
  morecomment=[s]{<?}{?>},
  morekeywords={xmlns,version,typex:AsyncRecords,x:Arguments,x:Boolean,x:Byte,x:Char,x:Class,x:ClassAttributes,x:ClassModifier,x:Code,x:ConnectionId,x:Decimal,x:Double,x:FactoryMethod,x:FieldModifier,x:Int16,x:Int32,x:Int64,x:Key,x:Members,x:Name,x:Object,x:Property,x:Shared,x:Single,x:String,x:Subclass,x:SynchronousMode,x:TimeSpan,x:TypeArguments,x:Uid,x:Uri,x:XData,Grid.Column,Grid.ColumnSpan,Click,ClipToBounds,Content,DropDownOpened,FontSize,Foreground,Header,Height,HorizontalAlignment,HorizontalContentAlignment,IsCancel,IsDefault,IsEnabled,IsSelected,Margin,MinHeight,MinWidth,Padding,SnapsToDevicePixels,Target,TextWrapping,Title,VerticalAlignment,VerticalContentAlignment,Width,WindowStartupLocation,Binding,Mode,OneWay,xmlns:x}
}

\lstdefinestyle{python}{
language=Python,
showspaces=false,
showtabs=false,
breaklines=true,
showstringspaces=false,
breakatwhitespace=true,
escapeinside={(*@}{@*)},
columns=fullflexible,
commentstyle=\color{greencomments},
keywordstyle=\color{bluekeywords}\bfseries,
stringstyle=\color{redstrings},
identifierstyle=\color{purpleidentifiers},
}

%defaults
\lstset{
basicstyle=\ttfamily\scriptsize ,
extendedchars=false,
numbers=left,
numberstyle=\ttfamily\tiny,
stepnumber=1,
tabsize=4,
numbersep=5pt
}
\addbibresource{../../.library/bibliography.bib}
\begin{document}
\chapter{Dataset and Results}
\label{ch:dataset}

%From Analysis:
This section describes how the data that was gathered in the previous steps will be analysed using the proper tools. The main goal is to see whether certain toxic users are only toxic when they are viewing a certain streamer or in every channel they visit. So whether viewers match their toxicity level to the channel. Another interesting question is to see if certain channels are more toxic than others and how this affects the toxicity levels of the users. 

\section{Streamers per User}
Twitch users like to view different games. And therefore watch to different Streamers. In this dataset there is a large amount of users that only watch a single streamer, but a fraction watches multiple streamers on a regular basis. In Figure \ref{fig:streamPerUser} the streamers per users are plotted on a logarithmic scale. Since most streamers use bots to moderate the channel and send messages to their users, they are present in almost all streams. Common bot names like "Xanbot" and "Moobot" are removed from the list as they are not real users. \\

\begin{figure}[h]
	\centering
	\includegraphics[width=0.7\textwidth]{StreamersPerUser.png}
	\caption{Streamers per User}
	\label{fig:streamPerUser}
\end{figure}

\noindent
\begin{minipage}{.5\textwidth}
  
\section{Wordcount}
An interesting way to analyze the dataset is to look at the wordcount, so counting how often words occur in the chat. Words that are expected to top the list are articles of the english language like "the". Team names of teams that participate in the tournaments are also seen very frequently. Because these words are of little value to our wordcount analysis, they are removed from the list. The first 25 words of this filtered list is shown in Table \ref{wordcounttabel}.\\
The one topping the list, "vac", is used often when impressive plays happen in game to express disbelief. This conclusion is made because VAC is the name off the anti-cheat system of Valve (Valve Anti-Cheat System). It is however seen as non toxic as it is no real accusation of cheating but a mere fun way of expressing that a player is doing well.
Other words that occur often are "na", "eu" and "usa", because in the league of legends community a rivalry between the EU (European Union) and NA (North America) is very common. These words are used as adjectives to express positive or negative thoughts about players, heroes or teams. Depending on the origin of the player that sent the message. For example if a viewer from North America says "What an EU player!", it usually means he thinks the player is not doing very well. While the "EU" adjective could be substituted for "NA" to make a positive comment about the player. Other expected occurrences are common swear words like "fuck", which is high in the list. \\


\end{minipage}% This must go next to `\end{minipage}`
\begin{minipage}{.5\textwidth}
\centering
\captionof{table}{Number of times words are used.}
\label{wordcounttabel}
\begin{tabular}{|l|l|}
\hline
Word    & Wordcount \\ \hline
vac     & 134034    \\ \hline
lol     & 122647    \\ \hline
na      & 86260     \\ \hline
rip     & 46703     \\ \hline
god     & 40136     \\ \hline
nip     & 39178     \\ \hline
ez      & 37953     \\ \hline
bot     & 36822     \\ \hline
lg      & 31513     \\ \hline
drop    & 30235     \\ \hline
sourpls & 28612     \\ \hline
vp      & 27679     \\ \hline
eu      & 25986     \\ \hline
game    & 22379     \\ \hline
rekt    & 22321     \\ \hline
usa     & 22265     \\ \hline
nt      & 20692     \\ \hline
cobble  & 18345     \\ \hline
nice    & 18161     \\ \hline
good    & 17439     \\ \hline
fuck    & 17424     \\ \hline
why     & 17364     \\ \hline
faze    & 16938     \\ \hline
lmao    & 16266     \\ \hline
\end{tabular}

\end{minipage}

\section{Deleted messages}

Twitch also has its own filtering, which is done before messages appear in the chat. Some of the messages sent are therefore deleted by Twitch, as they view it as toxic. Deleted messages still appear in the chat log. However, the message itself is removed if the deleted flag is set to true. This way we can still analyze some of the data by just looking if the deleted flag is set to true, which is an indicator that a message might be toxic.\\
Points of interest about the deleted messages are:
\begin{itemize}
	\item Ratio of deleted messages per user
	\item Ratio of deleted messages per stream
\end{itemize}


\subsection{Deleted messages per user}
We wanted to see if a users behaves the same way in different video's. If this is the case, one could conclude that the toxicity is due the user and not the streamer. If the looks random, one could conclude that the streamer has more influence to the toxicity of the users.

However, this is very hard to visualize and generalize for different users/streamers of the entire dataset.
Because of this, we took a one streamer (AmazHS) and picked some videos and compared them.

\begin{verbatim}
[('datguyzed', 0.416), ('everdreen', 0.181)]
[('awildchocobo', 0.727), ('rommikoira', 0.571), ('maedalislie', 0.428)]
[('leckotut', 0.166)]
[('koran_mekka', 0.5)]
[('sokkimhong', 0.833), ('awesomememesspammedquick', 0.545), ('moonmoonderp', 0.5)]
[('jroblul', 0.636), ('silentdropx', 0.615), ('paskajaakko420', 0.4375)]
[('yoshinami', 0.454)]
[('generalkkona', 0.416), ('quietguy89', 0.207), ('srbombarder', 0.190)]
[('leckotut', 0.666), ('matt_friday', 0.12)]
[('comi_', 0.428)]
[('everdreen', 0.2)]
[('m9za', 0.714), ('sahrgeand', 0.692), ('veelyo', 0.111), ('drsmoke100', 0.091)]
[('leecolas', 0.307), ('gunnervine', 0.176), ('nielsnice', 0.133), ('lockiez', 0.066)]
[('meezy_money', 0.384)]
\end{verbatim}

As can be seen, there isn't a pattern visible. So it looks that the users are randomly being toxic.\\
However, it's very likely that we have not enough data to get such results. It's possible that we don't have the same users watching, and that the toxic users only appear in one video.

\subsection{Deleted messages per stream}
Before we calculate the ratio, we want to filter the data some more. We exclude all the users in a single chatlog that have posted less than 10 messages. So only users that posted more than 10 messages are considered 'real' users, that tend to watch the stream. 
We sum all ratio's of the users in a stream and calculate the average ratio. Then we plot the ratio against the amount of users with that ratio in figure \ref{fig:deletedPerStream}. The amount of users are normalized, so we can compare different sizes of streamers.
In figure \ref{fig:deletedPerStream} we took the three largest streamers in our dataset. Note that the largest streamer (MLG) has around 35000 users that had a message deleted. The second (eleaguetv) has around 10000 users and the third (dreamhackcs) only around 700.

\begin{figure}[h]
	\centering
	\includegraphics[width=0.7\textwidth]{DeletedPerStreamer.png}
	\caption{Ratio of deleted messages in a stream}
	\label{fig:deletedPerStream}
\end{figure}

As can be seen, the MLG stream has the most amount of low ratio users, and the amount decreases as the ratio increases. However, in the dreamhackcs data we see that the lowest ratio doesn't have the highest amount of users. The line increases in the beginning and the most users with deleted message have a ratio around 10-20$\%$. After this peak all streams show a similar pattern toward a higher ratio.
However, and the end we see a jump at a ratio of 1. These are the users that got all of their messages deleted (10+ messages).\\

We can also compare the total amount of messages in a channel, and compare that to the average ratio in the entire channel. We see that the bigger the streamer, the higher the ratio. However the ratio's are very close to eachother, and there are a lot of other factors (game, time of day, etc) involved to say that the bigger the channel, the more toxic it becomes. In the appendix is the full list for all streamers.

\begin{table}[]
\centering
\caption{Ratio and message data per streamer}
\label{my-label}
\begin{tabular}{l|lll}
Streamer 		 & Deleted messaged & Total messages 	& Ratio 	\\
mlg              & 301688 			& 2628092 			& 0.1148   	\\
eleaguetv        & 60761  			& 779918  			& 0.0779   	\\
dreamhackcs      & 8190   			& 112943  			& 0.0725	\\
\end{tabular}
\end{table}



\end{document}